%\documentclass[11pt,a4paper,fleqn,english]{memoir}
\documentclass[11pt,a4paper,fleqn,english,draft]{memoir}
\usepackage[T1]{fontenc}
\usepackage[utf8]{inputenc}
\setcounter{secnumdepth}{3}
\setcounter{tocdepth}{3}
\usepackage{amstext}
\usepackage{amssymb}

\usepackage[bookmarks,colorlinks=true,final]{hyperref}

\usepackage{natbib}

\makeatletter

%%%
%%% Aus postdiploma.sty:
%%%
\newif\ifchaptercolors
\chaptercolorstrue

\usepackage[final]{microtype}

% A4 paper size
\settrimmedsize{297mm}{210mm}{*}
\setlength{\trimtop}{0pt}
\setlength{\trimedge}{\stockwidth}
\addtolength{\trimedge}{-\paperwidth}

\providecommand*{\isopage}[1][9]{%
  \spinemargin=\paperwidth
  \divide\spinemargin #1\relax
  \uppermargin=\paperheight
  \divide\uppermargin #1\relax
  \setlrmarginsandblock{\spinemargin}{*}{2}
  \setulmarginsandblock{\uppermargin}{*}{2}}
%\isopage[9]
\providecommand*{\heinerfolderpage}{%
  \spinemargin=\paperwidth
  \divide\spinemargin 6\relax
  \foremargin=\paperwidth
  \divide\foremargin 7\relax
  \setlrmarginsandblock{\spinemargin}{\foremargin}{*}
  \uppermargin=\paperheight
  \divide\uppermargin 12\relax
  \setulmarginsandblock{\uppermargin}{*}{1}}
\heinerfolderpage
\setheaderspaces{1.5cm}{*}{1}
\checkandfixthelayout

\usepackage{color}
\definecolor{halfgray}{gray}{0.55}
\definecolor{webgreen}{rgb}{0,.5,0}
\definecolor{webbrown}{rgb}{.6,0,0}
\definecolor{Maroon}{cmyk}{0, 0.87, 0.68, 0.32}
\definecolor{RoyalBlue}{cmyk}{1, 0.50, 0, 0}
%\definecolor{Black}{cmyk}{0, 0, 0, 0}
\definecolor{black}{rgb}{0, 0, 0}

\definecolor{red2}{cmyk}{0, 1, 1, 0.07}
\definecolor{red3}{cmyk}{0, 1, 1, 0.2}
\definecolor{darkred}{cmyk}{0, 1, 1, 0.45}

% Wrapping headers not indented (Bringhurst)
\renewcommand{\@hangfrom}[1]{\noindent{#1}}

\usepackage{textcase}

\setsecheadstyle{\large\raggedright\setSpacing{0.95}\scshape\MakeTextUppercase}
\setsubsecheadstyle{\large\centering\itshape}
\setsubsubsecheadstyle{\normalsize\raggedright\itshape}

\setparaheadstyle{\normalsize\raggedright\scshape}
\setsubparaheadstyle{\normalsize\raggedright\scshape}

\setlength\mathindent{\parindent}

\copypagestyle{heiner}{plain}

\ifdraftdoc
  \makeoddfoot{heiner}%
              {}{\textsc{\thepage}}{\footnotesize [compiled
              \printtime\ on \the\day.\,\the\month.\,\the\year]}
\else
  \makeoddfoot{heiner}%
              {}{\textsc{\thepage}}{}
\fi
\makeevenfoot{heiner}%
  {}{\textsc{\thepage}}{}

\copypagestyle{heinerplain}{plain}

\aliaspagestyle{chapter}{heinerplain}
\pagestyle{heiner}

\renewcommand{\footnoterule}{}
\setlength{\footmarkwidth}{-1.0em}
\setlength{\footmarksep}{-\footmarkwidth}
\footmarkstyle{#1~}
\renewcommand*{\@makefnmark}{\hbox{\textsuperscript{\@thefnmark}}}
\setlength{\footnotesep}{4.2ex}

% Memoir doesn't play well with setspace, but has this build in
\setSpacing{1.2}

\let\stdforall\forall
\renewcommand{\forall}{\lower0.5ex\hbox{$\stdforall$}}

% Vgl. Bringhurst
\newcommand{\hairspace}{\kern .04167em}

\microtypesetup{expansion=false}%
\DeclareRobustCommand{\spacedallcaps}[1]{\textls[90]{\MakeTextUppercase{#1}}}%
\DeclareRobustCommand{\spacedlowsmallcaps}[1]{\textls[80]{\scshape\MakeTextLowercase{#1}}}%

% Disable single lines at the start of a paragraph (Schusterjungen)
\clubpenalty = 10000
% Disable single lines at the end of a paragraph (Hurenkinder)
\widowpenalty = 10000
\displaywidowpenalty = 10000 % formulas


\counterwithout{section}{chapter}

\setsecnumdepth{section}
\maxsecnumdepth{section}

\makechapterstyle{postdiploma}{%
  \setlength{\beforechapskip}{10ex}
  \setlength{\afterchapskip}{7ex}
  \def\chapterheadstart{\phantomsection\vspace*{\beforechapskip}}

  \renewcommand{\chapnumfont}{\LARGE\bfseries}
  \renewcommand{\printchaptername}{\centering}
  \renewcommand{\chapternamenum}{}
  \renewcommand{\chapnamefont}{\normalfont\large}
  \renewcommand{\printchapternum}{\centering {\chapnumfont\sectionsymbol\oldstylenums{\thechapter}}}
  \renewcommand{\afterchapternum}{\normalsize\\[1ex]}
  \renewcommand{\printchaptertitle}[1]{\centering {\chapnamefont\spacedallcaps{##1}}}
  \renewcommand{\afterchaptertitle}{\par\nobreak\vskip \afterchapskip}
}
\chapterstyle{postdiploma}

\newcommand{\addchaptertotoc}[1]{%
  \addcontentsline{toc}{chapter}{%
    \texorpdfstring{\MakeLowercase{#1}}{#1}}}
\newcommand{\diplomachapter}[1]{%
  \chapter*{#1}%
  \addchaptertotoc{#1}}
\newcommand{\diplomachapterandtoc}[2]{%
  \chapter*{#1}%
  \addcontentsline{toc}{chapter}{\texorpdfstring{#2}{#1}}}

\newcommand{\sectionandtoc}[2]{%
  \section*{#1}%
  \addcontentsline{toc}{section}{\texorpdfstring{#2}{#1}}}

\setlength{\epigraphwidth}{0.8\linewidth}
\epigraphposition{flushleft}
\epigraphsourceposition{flushleft}
\setlength{\epigraphrule}{0pt}
\epigraphfontsize{\footnotesize}

%%% From memmanadd, PDF-page 41
\renewcommand*{\cftchapterfillnum}[1]{%
  {\cftchapterleader}\nobreak
  \hbox to 1.5em{\cftchapterpagefont \hfil#1}\cftchapterafterpnum\par}
\renewcommand*{\cftsectionfillnum}[1]{%
  {\cftsectionleader}\nobreak
  \hbox to 1.5em{\cftsectionpagefont \hfil#1}\cftsectionafterpnum\par}

%% memmanadd page 22 (PDF page 40)
\ifchaptercolors
  \renewcommand{\cftchapterfont}{\hfill\color{red3}\scshape}
\else
  \renewcommand{\cftchapterfont}{\hfill\scshape}
\fi
%\renewcommand{\cftchapterleader}{ \textperiodcentered\space}
\renewcommand{\cftchapterleader}{\space}
\renewcommand{\cftchapterpagefont}{\normalfont\scshape}
\renewcommand{\cftchapterafterpnum}{\cftparfillskip}
\setpnumwidth{0em}

\renewcommand{\cftsectionfont}{\hfill\normalfont}
\renewcommand{\cftsectionleader}{\space}
\renewcommand{\cftsectionpresnum}{\bfseries\scshape}
\renewcommand{\cftsectionpagefont}{\normalfont\scshape}
\renewcommand{\cftsectionafterpnum}{\cftparfillskip}
\setlength{\cftsectionnumwidth}{1em}

\renewcommand{\contentsname}{Inhalt}
\renewcommand{\bibname}{Literatur}

\setrmarg{1.5em}

%%%
%%% Aus mathmacs.sty:
%%%
\usepackage{amsmath}
\usepackage{amssymb}

\newif\iftheoremcolors
\theoremcolorsfalse

% Lemma-, Satz-, usw. Umgebungen
  \usepackage{amsthm}
  \swapnumbers

  % For color redefined from amsthm.sty
  \def\th@remark{%
    \iftheoremcolors
      \thm@headfont{\color{Maroon}\itshape}%
      \thm@notefont{\/\upshape\color{black}}%
      \thm@headpunct{\color{black}.}%
    \else
      \thm@headfont{\bfseries}%
      \thm@notefont{\normalfont}%
      \thm@headpunct{\bfseries.}%
    \fi
    \normalfont % body font
    \thm@preskip\topsep \divide\thm@preskip\tw@
    \thm@postskip\thm@preskip}

  %\zerotrivseps % Vgl memman, Seite 103 (PDF-Seite 137)
  %\newlength{\proofskip}
  %\setlength{\proofskip}{2pt plus 1pt minus 1pt}
  %% \renewenvironment{proof}[1][\proofname]{\par
  %%   \pushQED{\qed}%
  %%   \normalfont \vspace{-\topsep}\vspace{-\partopsep}\vspace{\proofskip}
  %%   \relax
  %%   \trivlist\tightlist
  %% \item[\hskip\labelsep
  %%   \itshape
  %%   #1\@addpunct{.}]\ignorespaces
  %% }{%
  %%   \popQED\endtrivlist\@endpefalse
  %% }

  \theoremstyle{remark}
  %\theoremstyle{plain} % Prof. Voigts Wunsch
  %\newtheorem{theorem}{Theorem}[chapter]
  \newtheorem{theorem}{Theorem}[section]
  \newtheorem*{theorem*}{Theorem}

  \newtheorem{lemma}[theorem]{Lemma}
  \newtheorem*{lemma*}{Lemma}

  \newtheorem{proposition}[theorem]{Proposition} % Hilfssatz?
  \newtheorem*{proposition*}{Proposition}        %          ?

  \newtheorem{corollary}[theorem]{Corollary}
  \newtheorem*{corollary*}{Corollary}

  \theoremstyle{remark}
  \newtheorem*{definition}{Definition}
  \newtheorem*{remark}{Remark}
  \newtheorem*{remarks}{Remarks}

  %\renewcommand{\qedsymbol}{//}
  %\renewcommand{\qed}{\hfill\mbox{\raggedright\rule{0.25em}{0.5\baselineskip}}}

  \swapnumbers
  \let\@upn\relax
  \newtheorem{nosectheorem}{Theorem}
  \newtheorem{noseclemma}[nosectheorem]{Lemma}
  \newtheorem{nosecproposition}[nosectheorem]{Proposition}

  %\newcommand{\thetheoremwithdot}{\alph{theorem}}

\addtodef{\appendix}{}{%
  \let\theorem\nosectheorem%
  \let\lemma\noseclemma%
  \let\proposition\nosecproposition
  %\let\thetheorem\thetheoremwithdot%
}


%
% Special math macros
%

% from gkpmac.tex, line 720--723
%\def\_#1{\def\next{#1}%
% \ifx\next\risingsign\expandafter\rising\else^{\underline{#1}}\fi}
%\def\risingsign{^}
%\def\rising#1{^{\overline{#1}}}

% from amsldoc, page 17.
\providecommand{\abs}[1]{\lvert#1\rvert}
\providecommand{\norm}[1]{\lVert#1\rVert}
% von mir ausgedacht :). Oder Fall (1) mit \bigg? Vgl. amsldoc, PDF-Seite 20
\providecommand{\Abs}[1]{
  \mathchoice{\Bigl\lvert#1\Bigr\rvert}%
             {\bigl\lvert#1\bigr\rvert}%
             {\lvert#1\rvert}%
             {\lvert#1\rvert}}

% This one with optional argument for indices. Extra { } doesn't work
% since the contents are potentially bigger.
\providecommand{\Norm}[2][]{
  \mathchoice{\Bigl\lVert#2\Bigr\rVert_{#1}}%
             {\bigl\lVert#2\bigr\rVert_{#1}}%
             {\lVert#2\rVert_{#1}}%
             {\lVert#2\rVert_{#1}}}

% Hand selection. (Experimental)
\providecommand{\bigabs}[1]{\bigl\lvert#1\bigr\rvert}
\providecommand{\Bigabs}[1]{\Bigl\lvert#1\Bigr\rvert}
\providecommand{\biggabs}[1]{\biggl\lvert#1\biggr\rvert}
\providecommand{\Biggabs}[1]{\Biggl\lvert#1\Biggr\rvert}

\providecommand{\bignorm}[1]{\bigl\lVert#1\bigr\rVert}
\providecommand{\Bignorm}[1]{\Bigl\lVert#1\Bigr\rVert}
\providecommand{\biggnorm}[1]{\biggl\lVert#1\biggr\rVert}
\providecommand{\Biggnorm}[1]{\Biggl\lVert#1\Biggr\rVert}


% The same for sets
\providecommand{\set}[1]{\{#1\}}
\providecommand{\Set}[1]{
  \mathchoice{\bigl\{#1\bigr\}}%
             {\{#1\}}%
             {\{#1\}}%
             {\{#1\}}}
\providecommand{\bigset}[1]{\bigl\{#1\bigr\}}
\providecommand{\Bigset}[1]{\Bigl\{#1\Bigr\}}
\providecommand{\biggset}[1]{\biggl\{#1\biggr\}}
\providecommand{\Biggset}[1]{\Biggl\{#1\Biggr\}}

% The same for parentheses and brackets
\providecommand{\parens}[1]{(#1)}
%\providecommand{\Parens}[1]{
%  \mathchoice{\bigl(#1\bigr)}%
%             {(#1)}%
%             {(#1)}%
%             {(#1)}}
\providecommand{\bigparens}[1]{\bigl(#1\bigr)}
\providecommand{\Bigparens}[1]{\Bigl(#1\Bigr)}
\providecommand{\biggparens}[1]{\biggl(#1\biggr)}
\providecommand{\Biggparens}[1]{\Biggl(#1\Biggr)}
\let\Parens\bigparens

\providecommand{\brackets}[1]{[#1]}
%\providecommand{\Brackets}[1]{
%  \mathchoice{\bigl[#1\bigr]}%
%             {[#1]}%
%             {[#1]}%
%             {[#1]}}
\providecommand{\bigbrackets}[1]{\bigl[#1\bigr]}
\providecommand{\Bigbrackets}[1]{\Bigl[#1\Bigr]}
\providecommand{\biggbrackets}[1]{\biggl[#1\biggr]}
\providecommand{\Biggbrackets}[1]{\Biggl[#1\Biggr]}
\let\Brackets\bigbrackets

% Way easier to type (idea of Dr. H. Vogt)
\let\<\langle
\let\>\rangle

\providecommand{\angles}[1]{\<#1\>}
\providecommand{\bigangles}[1]{\bigl\<#1\bigr\>}
\providecommand{\Bigangles}[1]{\Bigl\<#1\Bigr\>}
\providecommand{\biggangles}[1]{\biggl\<#1\biggr\>}
\providecommand{\Biggangles}[1]{\Biggl\<#1\Biggr\>}
\let\Angles\bigangles

% Angular brackets pair
\providecommand{\apair}[2]{\<#1,#2\>}
\providecommand{\Apair}[2]{
  \mathchoice{\Bigl\<#1,#2\Bigr\>}%
             {\bigl\<#1,#2\bigr\>}%
             {\<#1,#2\>}%
             {\<#1,#2\>}}

\providecommand{\bigmid}{\bigm\vert}
\providecommand{\Bigmid}{\Bigm\vert}
\providecommand{\biggmid}{\biggm\vert}
\providecommand{\Biggmid}{\Biggm\vert}


\DeclareMathOperator{\lin}{lin}          % Lineare Hülle
\DeclareMathOperator{\sgn}{sgn}          % Signum
\DeclareMathOperator{\id}{id}            % Identität
\DeclareMathOperator{\grad}{grad}        % Gradient
\DeclareMathOperator{\diam}{diam}        % Diameter
\DeclareMathOperator{\spt}{spt}          % Support
\DeclareMathOperator{\supp}{supp}          % Support
\DeclareMathOperator*{\esssup}{ess\,sup} % Wesentliches Supremum
\DeclareMathOperator{\dist}{dist}        % Distance

\DeclareMathOperator{\ran}{ran}          % Range
\DeclareMathOperator{\tr}{tr}            % Trace
\DeclareMathOperator{\Borel}{Borel}      % Borel sets

\newcommand{\spec}{\sigma}               % Spectrum
\newcommand{\resolv}{\varrho}            % Resolvent set

%\newcommand{\1}{\textrm{1}}              % Indikatorfunktion
%\newcommand{\1}{\chi}                    % Indikatorfunktion von Peter
\newcommand{\dotid}{\,\cdot\,}           % Hier-Einsetzen-Identität

%% http://tex.stackexchange.com/questions/488/blackboard-bold-characters
%\DeclareSymbolFont{bbold}{U}{bbold}{m}{n}
%\DeclareSymbolFontAlphabet{\mathbbold}{bbold}
%\newcommand{\1}{\mathbbold{1}}
\newcommand{\1}{1}

\let\textdef\textit
\newcommand{\where}[1]{
  \mathchoice{\qquad(#1)}%
             {\,\ (#1)}%     % Dr. Vogt version
             {\,(#1)}%       % Last two really shouldn't be used
             {\,(#1)}}
\newcommand{\bigwhere}[1]{
  \mathchoice{\qquad\bigl(#1\bigr)}%
             {\,\ \bigl(#1\bigr)}%     % Dr. Vogt version
             {\,(#1)}%       % Last two really shouldn't be used
             {\,(#1)}}


% A complement to \smash, \llap, and \rlap. TUGboat 22 (2001), 350--352.
% See http://math.arizona.edu/~aprl/publications/mathclap/
\def\clap#1{\hbox to 0pt{\hss#1\hss}}
\def\mathllap{\mathpalette\mathllapinternal}
\def\mathrlap{\mathpalette\mathrlapinternal}
\def\mathclap{\mathpalette\mathclapinternal}
\def\mathllapinternal#1#2{%
           \llap{$\mathsurround=0pt#1{#2}$}}
\def\mathrlapinternal#1#2{%
           \rlap{$\mathsurround=0pt#1{#2}$}}
\def\mathclapinternal#1#2{%
           \clap{$\mathsurround=0pt#1{#2}$}}


% Abschluss, Inneres
\providecommand{\cl}[1]{\overline{#1}}
\DeclareMathOperator{\interior}{int}

% amsldoc, page 22 (PDF page 26)
\providecommand{\field}[1]{\mathbb{#1}}

% Normales \mathbb _scheint_ zu klein für die anderen Symbole
%\providecommand{\field}[1]{%
%  {\mathchoice{\mbox{\small$\mathbb{#1}$}}%
%              {\mbox{\small$\mathbb{#1}$}}%
%              {\mbox{\large$\scriptscriptstyle\mathbb{#1}$}}%
%              {\mbox{\footnotesize$\scriptscriptstyle\mathbb{#1}$}}}}

\providecommand{\N}{\field{N}}
\providecommand{\Z}{\field{Z}} % true fields following now :)
\providecommand{\Q}{\field{Q}}
\providecommand{\R}{\field{R}}
\providecommand{\C}{\field{C}}
\providecommand{\K}{\field{K}} % generic Field == Körper

\DeclareMathOperator{\TextRe}{Re}
\DeclareMathOperator{\TextIm}{Im}
\renewcommand{\Re}{\TextRe}
\renewcommand{\Im}{\TextIm}

% interior, with beauty (from yhmath, but hacked by me).
\DeclareSymbolFont{yhmathsymbols}{OMX}{yhex}{m}{n}
\DeclareMathAccent{\wideparen}{\mathord}{yhmathsymbols}{"F3}
\edef\@tempa#1#2{\def#1{\mathaccent\string"\noexpand\accentclass@#2 }}%
\@tempa\ring{006} % that number is correct for MinionPro; was: 017.
\newcommand{\widering}[1]{\smash{\ensuremath{\ring{\wideparen{#1}}}}}

\newcommand{\A}{\mathcal{A}}
\newcommand{\B}{\mathcal{B}}
\newcommand{\F}{\mathcal{F}}
\newcommand{\G}{\mathcal{G}}
\newcommand{\I}{\mathcal{I}}
\newcommand{\J}{\mathcal{J}}
\let\suppressedL\L
\renewcommand{\L}{\mathcal{L}}
\newcommand{\Zz}{\mathcal{Z}}
\newcommand{\M}{\mathcal{M}}
\newcommand{\Nn}{\mathcal{N}}
\newcommand{\Cee}{\mathcal{C}}
\newcommand{\E}{\mathcal{E}}

\let\longHungarianUmlaut\H
\renewcommand{\H}{\mathcal{H}}

\newcommand{\Pot}{\mathfrak{P}}

\newcommand{\T}{\mathcal{T}}
\let\sectionsymbol\S
\DeclareInputText{167}{\sectionsymbol} % for latin1
\renewcommand{\S}{\mathcal{S}}
\newcommand{\U}{\mathcal{U}}

\newcommand{\Beta}{\mathrm{B}}
\newcommand{\Rho}{\mathrm{P}}

% Residue
\DeclareMathOperator*{\Res}{Res}

% Small and big O
\newcommand{\Oh}{\mathrm{O}}
\newcommand{\oh}{\mathrm{o}}

% Airy function
\DeclareMathOperator{\Ai}{Ai}

% "Such that" für Mengen
\providecommand{\st}{;\,}

% "d" für Integrale. Roman?
\newcommand{\upd}{\mathrm{d}}
\let\underdot\d
%\renewcommand{\d}{\,\upd} % Mathematiker-Version
\renewcommand{\d}{\upd}   % Physiker-Version
\newcommand{\dx}{\d x}
\newcommand{\dt}{\d t}
\newcommand{\ds}{\d s}
\newcommand{\dy}{\d y}
\newcommand{\dr}{\d r}
\newcommand{\dz}{\d z}
% Ghosts of departed quantities
\newcommand{\dbyd}[1]{\frac{\upd}{\upd{#1}}}

\newcommand{\fin}{\mathrm{fin}}
\newcommand{\ac}{\mathrm{ac}}
\newcommand{\loc}{\mathrm{loc}}
\newcommand{\HS}{\mathrm{HS}}

\newcommand{\textimplies}{\ensuremath{\,\Rightarrow\,}}

\DeclareMathOperator{\im}{im}
\newcommand{\PV}{\,\textsc{pv}\!\!}

\DeclareMathOperator{\vol}{vol}

\newcommand{\upto}{\uparrow}
\newcommand{\downto}{\downarrow}

\newcommand{\sto}{\xrightarrow{\text{s}}}
\newcommand{\wto}{\xrightarrow{\text{w}}}

% Für Hacks: Die breite einer Formelzeile
\newlength{\mathdiplaywidth}
\setlength{\mathdiplaywidth}{\textwidth}
\addtolength{\mathdiplaywidth}{-\mathindent}
\addtolength{\mathdiplaywidth}{-\columnsep}
%\setlength{\mathdiplaywidth}{324.37518pt}

\newcommand{\sigmaAlg}{\sigma}
\newcommand{\from}{\colon}
\newcommand{\floor}[1]{\lfloor #1\rfloor}

\newcommand{\rightbar}{\bigr\rvert}
\newcommand{\Bigrightbar}{\Bigr\rvert}
\newcommand{\biggrightbar}{\biggr\rvert}
\newcommand{\Biggrightbar}{\Biggr\rvert}
\let\oldrestriction\restriction
\renewcommand{\restriction}[1]{\rvert_{#1}}

\providecommand{\raisedversal}[1]{{\Huge\swshape{#1}}}
\midsloppy

\let\notto\nrightarrow

\counterwithout{section}{theorem}

\makeatother

\usepackage{babel}

\frenchspacing

\usepackage[final]{graphicx}
